\documentclass{article}
\usepackage{bm}
\usepackage{amsmath}
\usepackage{amssymb}
\usepackage{graphicx}
\usepackage[colorlinks=true,urlcolor=blue]{hyperref}
\usepackage{geometry}
\geometry{margin=1in}
\usepackage{multicol}
\usepackage{paralist}
\usepackage{todonotes}
\setlength{\marginparwidth}{2.15cm}
\usepackage{booktabs}
\usepackage{enumitem}
\graphicspath{{../}}
\usepackage{setspace}
\doublespacing

\newcommand{\norm}[1]{\left\lVert#1\right\rVert}
\newcommand\numberthis{\addtocounter{equation}{1}\tag{\theequation}}

\begin{document}

\section*{}
\begin{center}
  \centerline{\textsc{\LARGE Homework 4}}
  \vspace{0.5em}
  \centerline{\textsc{Graphical Models and Learning Theory}}
  \vspace{1em}
  \textsc{\large Dana Van Aken} \\
\end{center}

\section*{Problem 2: Learning Theory}

\subsection*{Problem 2.1: Learning Decision Lists}

\subsection*{Problem 2.2: VC Dimension}

\begin{enumerate}

\item Intervals in R: example is positive if it lies within the interval
\begin{itemize}

\item 1 point on the real line: \\
The possible sets of points are: + or -. + can be shattered by including it in the interval.
- can be shattered by not including it in the interval.

\item 2 points on the real line: \\
The possible sets of points are:  +  -   OR   -  +.   Both of these sets can be shattered by including the + point in the interval and
excluding the - point from the interval.

\item 3 points on the real line: \\


\end{itemize}

\end{enumerate}

\end{document}