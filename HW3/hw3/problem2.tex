\documentclass{article}
\usepackage{bm}
\usepackage{amsmath}
\usepackage{amssymb}
\usepackage{graphicx}
\usepackage[colorlinks=true,urlcolor=blue]{hyperref}
\usepackage{geometry}
\geometry{margin=1in}
\usepackage{multicol}
\usepackage{paralist}
\usepackage{todonotes}
\setlength{\marginparwidth}{2.15cm}
\usepackage{booktabs}
\usepackage{enumitem}
\graphicspath{{../}}
\usepackage{setspace}
\doublespacing

\newcommand{\norm}[1]{\left\lVert#1\right\rVert}

\begin{document}

\section*{}
\begin{center}
  \centerline{\textsc{\LARGE Homework 3}}
  \vspace{0.5em}
  \centerline{\textsc{Regression, Gaussian Processes, and Boosting}}
  \vspace{1em}
  \textsc{\large Dana Van Aken} \\
\end{center}

\section*{Problem 1: Gaussian Processes}

\begin{enumerate}[label=(\alph*)]
\setlength\itemsep{1em}

\item % (a)

\item % (b)

\item % (c)

\item % (d)

\item % (e)

\item % (f)

\end{enumerate}

\section*{Problem 2: Regression}

\subsection*{2.1 Why Lasso Works}

\begin{enumerate}
\setlength\itemsep{1em}

\item Write $J_\lambda(\beta)$ in the form $J_\lambda(\beta)=g(y)+\sum_1^df(X_i,y,\beta_i,\lambda)$, $\lambda > 0$: 

$J_\lambda(\beta)=\frac{1}{2}\norm{y-X\beta}^2+\lambda\norm{\beta}$ 

$=\frac{1}{2}(y-X\beta)^T(y-X\beta)+\lambda\norm{\beta}$ 

$=\frac{1}{2}[y^Ty-2y^TX\beta+(X\beta)^TX\beta]+\lambda\norm{\beta}$ 

$=\frac{1}{2}[y^Ty-2y^TX\beta+\beta^TX^TX\beta]+\lambda\norm{\beta}$ 

$=\frac{1}{2}[y^Ty-2y^TX\beta+\beta^T\beta]+\lambda\norm{\beta}$ \hfill $(X^TX=I)$ 

$=\frac{1}{2}y^Ty-y^TX\beta+\frac{1}{2}\beta^T\beta+\lambda\norm{\beta}$ 

$=\frac{1}{2}y^Ty+\sum_{i=1}^d\frac{1}{2}\beta_i^T\beta_i-y^TX_i\beta_i+\lambda\norm{\beta_i}$ 

Let $g(y)=\frac{1}{2}y^Ty$ and $f(X_i,y,\beta_i,\lambda)=\frac{1}{2}\beta_i^T\beta_i-y^TX_i\beta_i+\lambda\norm{\beta_i}$, then: 

$\bm{J_\lambda(\beta)=g(y)+\sum_{i=1}^df(X_i,y,\beta_i,\lambda)}$ 



\item $\beta_i^\ast>0$:

Calculating the derivative of $f(X_i,y,\beta_i^\ast,\lambda)$, (where $i$ in $\{1...d\}$), we get: 

$\frac{f(X_i,y,\beta_i^\ast,\lambda)}{d\beta_i^\ast}=\beta_i^\ast-y^TX_i+\lambda$

Setting the LHS equal to zero and solving for $\beta_i^\ast$ gives:

\begin{equation}
\bm{\beta_i^\ast=y^TX_i-\lambda}
\end{equation}


\item $\beta_i^\ast<0$:

Calculating the derivative of $f(X_i,y,\beta_i^\ast,\lambda)$, (where $i$ in $\{1...d\}$), we get: 

$\frac{f(X_i,y,\beta_i^\ast,\lambda)}{d\beta_i^\ast}=-\beta_i^\ast-y^TX_i+\lambda$

Setting the LHS equal to zero and solving for $\beta_i^\ast$ gives:

\begin{equation}
\bm{\beta_i^\ast=-y^TX_i+\lambda}
\end{equation}

\item % (4)

In both equations (1) and (2), as we increase $\lambda$, $\beta_i^\ast$ gets closer and closer to zero
(assuming $\beta_i^\ast$ and $y^TX_i$ are the same sign).
Once you increase $\lambda$ enough that $\beta_i^\ast$ reaches zero, it sticks there because moving it below zero
increases the L1 penalty and moves it further away from the least squares term (mathematically, $\lambda$ switches
its sign at this point because of the characteristics of the absolute value function). 

\item % (5)

Calculating the derivative of $f(X_i,y,\beta_i^\ast,\lambda)$, (where $i$ in $\{1...d\}$),
with the regularization term $\frac{1}{2}\norm{\beta_i^\ast}_2^2$ we get: 

$\frac{f(X_i,y,\beta_i^\ast,\lambda)}{d\beta_i^\ast}=\beta_i^\ast-y^TX_i+\lambda\beta_i^\ast$

Setting the LHS equal to zero and solving for $\beta_i^\ast$ gives:

$\beta_i^\ast=\frac{y^TX_i}{1+\lambda}$

Unlike equations (1) and (2), there is no value of alpha that can drive $\beta_i^\ast$ to zero.
This demonstrates why Lasso regression often results in ``sparser" solutions whereas Ridge regression
does not.

\end{enumerate}
\subsection*{2.2 Bayesian regression and Gaussian process}

\begin{enumerate}
\setlength\itemsep{1em}

\item % (1)

\begin{enumerate}

\item % (a)

\item % (b)

\end{enumerate}

\item % (2)

\item % (3)

\item % (4)

\end{enumerate}
\end{document}